\documentclass[10pt, letterpaper]{article}

% Packages:
\usepackage[
    ignoreheadfoot, % set margins without considering header and footer
    top=1.5 cm, % seperation between body and page edge from the top
    bottom=1 cm, % seperation between body and page edge from the bottom
    left=2 cm, % seperation between body and page edge from the left
    right=2 cm, % seperation between body and page edge from the right
    footskip=1.0 cm, % seperation between body and footer
    % showframe % for debugging 
]{geometry} % for adjusting page geometry
\usepackage{titlesec} % for customizing section titles
\usepackage{tabularx} % for making tables with fixed width columns
\usepackage{array} % tabularx requires this
\usepackage[dvipsnames]{xcolor} % for coloring text
\definecolor{primaryColor}{RGB}{0, 0, 0} % define primary color
\usepackage{enumitem} % for customizing lists
\usepackage{fontawesome5} % for using icons
\usepackage{amsmath} % for math
\usepackage[
    pdftitle={Loch Sandro's CV},
    pdfauthor={Sandro Loch},
    pdfcreator={LaTeX with RenderCV},
    colorlinks=true,
    urlcolor=primaryColor
]{hyperref} % for links, metadata and bookmarks
\usepackage[pscoord]{eso-pic} % for floating text on the page
\usepackage{calc} % for calculating lengths
\usepackage{bookmark} % for bookmarks
\usepackage{lastpage} % for getting the total number of pages
\usepackage{changepage} % for one column entries (adjustwidth environment)
\usepackage{paracol} % for two and three column entries
\usepackage{ifthen} % for conditional statements
\usepackage{needspace} % for avoiding page brake right after the section title
\usepackage{iftex} % check if engine is pdflatex, xetex or luatex

% Ensure that generate pdf is machine readable/ATS parsable:
\ifPDFTeX
    \input{glyphtounicode}
    \pdfgentounicode=1
    \usepackage[T1]{fontenc}
    \usepackage[utf8]{inputenc}
    \usepackage{lmodern}
\fi

\usepackage{charter}

% Some settings:
\raggedright
\AtBeginEnvironment{adjustwidth}{\partopsep0pt} % remove space before adjustwidth environment
\pagestyle{empty} % no header or footer
\setcounter{secnumdepth}{0} % no section numbering
\setlength{\parindent}{0pt} % no indentation
\setlength{\topskip}{0pt} % no top skip
\setlength{\columnsep}{0.15cm} % set column seperation
\pagenumbering{gobble} % no page numbering

\titleformat{\section}{\needspace{4\baselineskip}\bfseries\large}{}{0pt}{}[\vspace{1pt}\titlerule]

\titlespacing{\section}{
    % left space:
    -1pt
}{
    % top space:
    0.3 cm
}{
    % bottom space:
    0.2 cm
} % section title spacing

\renewcommand\labelitemi{$\vcenter{\hbox{\small$\bullet$}}$} % custom bullet points
\newenvironment{highlights}{
    \begin{itemize}[
        topsep=0.10 cm,
        parsep=0.10 cm,
        partopsep=0pt,
        itemsep=0pt,
        leftmargin=0 cm + 10pt
    ]
}{
    \end{itemize}
} % new environment for highlights


\newenvironment{highlightsforbulletentries}{
    \begin{itemize}[
        topsep=0.10 cm,
        parsep=0.10 cm,
        partopsep=0pt,
        itemsep=0pt,
        leftmargin=10pt
    ]
}{
    \end{itemize}
} % new environment for highlights for bullet entries

\newenvironment{onecolentry}{
    \begin{adjustwidth}{
        0 cm + 0.00001 cm
    }{
        0 cm + 0.00001 cm
    }
}{
    \end{adjustwidth}
} % new environment for one column entries

\newenvironment{twocolentry}[2][]{
    \onecolentry
    \def\secondColumn{#2}
    \setcolumnwidth{\fill, 4.5 cm}
    \begin{paracol}{2}
}{
    \switchcolumn \raggedleft \secondColumn
    \end{paracol}
    \endonecolentry
} % new environment for two column entries

\newenvironment{threecolentry}[3][]{
    \onecolentry
    \def\thirdColumn{#3}
    \setcolumnwidth{, \fill, 4.5 cm}
    \begin{paracol}{3}
    {\raggedright #2} \switchcolumn
}{
    \switchcolumn \raggedleft \thirdColumn
    \end{paracol}
    \endonecolentry
} % new environment for three column entries

\newenvironment{header}{
    \setlength{\topsep}{0pt}\par\kern\topsep\centering\linespread{1.5}
}{
    \par\kern\topsep
} % new environment for the header

\newcommand{\placelastupdatedtext}{% \placetextbox{<horizontal pos>}{<vertical pos>}{<stuff>}
  \AddToShipoutPictureFG*{% Add <stuff> to current page foreground
    \put(
        \LenToUnit{\paperwidth-2 cm-0 cm+0.05cm},
        \LenToUnit{\paperheight-1.0 cm}
    ){\vtop{{\null}\makebox[0pt][c]{
        \small\color{gray}\textit{Last updated in September 2024}\hspace{\widthof{Last updated in September 2024}}
    }}}%
  }%
}%

% save the original href command in a new command:
\let\hrefWithoutArrow\href

% new command for external links:


\begin{document}
    \newcommand{\AND}{\unskip
        \cleaders\copy\ANDbox\hskip\wd\ANDbox
        \ignorespaces
    }
    \newsavebox\ANDbox
    \sbox\ANDbox{$|$}

    \begin{header}
        \fontsize{25 pt}{25 pt}\selectfont Sandro Loch

        \vspace{5 pt}

        \normalsize
        \kern 5.0 pt%
        \mbox{\hrefWithoutArrow{mailto:es.loch@gmail.com}{es.loch@gmail.com}}%
        \kern 5.0 pt%
        \AND%
        \kern 5.0 pt%
        \mbox{\hrefWithoutArrow{https://www.linkedin.com/in/esloch/}{linkedin.com/in/esloch}}%
        \kern 5.0 pt%
        \AND%
        \kern 5.0 pt%
        \mbox{\hrefWithoutArrow{https://github.com/esloch}{github.com/esloch}}%

        \AND%
        \kern 5.0 pt%
        \mbox{\hrefWithoutArrow{tel:+55-47-98455-69-32}{55-47-98455-6932}}%
        \kern 5.0 pt%
       
        \mbox{Santa Catarina - Brazil}%
        \kern 5.0 pt%
        \AND%
        
    \end{header}

    \vspace{5 pt - 0.3 cm}


    \section{Summary}
        
        \begin{onecolentry}
            \href{https://rendercv.com}I am a Full-Stack DevOps Developer with over five years of experience designing and implementing scalable backend systems, automating infrastructure, and contributing to impactful open-source projects. My work focuses on developing robust solutions for data-intensive applications, ensuring high availability, performance, and automation throughout the development lifecycle.
        \end{onecolentry}

        \vspace{0.2 cm}
   
    \section{Experience (last 5 years)}
        
        \begin{twocolentry}{
            12/2023 -- present
        }
            \textbf{Software Engineer}, OSL - Open Science Labs / LiteRev - Université de Genève\end{twocolentry}

        \vspace{0.10 cm}
        \begin{onecolentry}
            \begin{highlights}
                \item Developed and optimized backend architecture using Docker, NGINX, Django, Celery, Redis, and Elasticsearch for automating scientific data collection and processing. \\
                \item Developed and integrated Retrieval-Augmented Generation (RAG) workflows to improve AI-driven document retrieval.\\
                \item Implemented CI/CD pipelines with GitHub Actions for seamless deployment and automated testing. \\
                \item Managed SSL certificates and configured NGINX servers for secure and efficient access. \\
                \item Developed automated pipelines for collecting and processing scientific articles from sources like arXiv, MedRxiv, and PubMed Central using APIs and Django. \\
                \item Integrated Elasticsearch for efficient data indexing and retrieval, enhancing search performance. \\
                \item Led migration and configuration of virtual servers, ensuring optimal PostgreSQL deployment. \\
            \end{highlights}
        \end{onecolentry}

        \vspace{0.2 cm}

        \begin{twocolentry}{
            03/2019 -- 12/2024
        }
            \textbf{Full-Stack \band Backend Developer}, Fiotec - Scientific and Technological Development in Health Foundation\end{twocolentry}

        \vspace{0.10 cm}
        \begin{onecolentry}
            \begin{highlights}
                \item Developed and maintained Django applications for reporting dengue and other arboviruses cases. \\
                \item Created and maintained PostgreSQL databases, including modeling, massive data loading, and query optimization. \\
                \item Developed APIs with Django Rest Framework for querying epidemiological data. \\
                \item Optimized the development pipeline using Ansible, Docker, Celery, PostgreSQL, and CI/CD (GitHub Actions). \\
                \item Managed large volumes of data with Pandas and Dask, producing interactive reports and analyses. \\
                \itemmarker Automated climate data ingestion by integrating APIs and processing data using Python, Pandas, and SQLAlchemy. \\
                \item Implemented PostgreSQL replication using Bucardo, ensuring high availability of epidemiological data. \\
                \item Created scripts and playbooks to synchronize and validate historical alerts in PostgreSQL with Ansible automation. \\
                \item Deployed scalable object storage solutions with MinIO and Docker for epidemiological data repositories. \\
            \end{highlights}
        \end{onecolentry}
    \section{Technologies}
        
        \begin{onecolentry}
            \begin{highlights}
        
                \item \textbf{Programming and Scripting:} Python, SQL, Bash
                \item \textbf{Web Development and APIs:} Django, Django Rest Framework, Django ORM, Bootstrap
                \item \textbf{Data Management and Analysis:} PostgreSQL, SQLAlchemy, Bucardo, MinIO, Pandas, Dask, Ibis
                \item \textbf{DevOps and Automation:} Docker, Ansible, GitHub Actions, Celery, APScheduler
                \item \textbf{Networking and Infrastructure:} Linux Servers, DNS, DHCP, Group Policy, Hybrid PABX Systems
                \item \textbf{Visualization and Reporting:} Plotly, Matplotlib, Highcharts
                \item \textbf{Package Management:} Conda, Poetry, Pip
                \item \textbf{CI/CD and Testing:} GitHub Actions, deployment pipelines, unit testing
                \item \textbf{Tools and Technologies:} Elasticsearch, Redis, NGINX, Let’s Encrypt
                \item \textbf{Security & Compliance:} ITIL best practices, data integrity, and process automation  
                
            \end{highlights}
        \end{onecolentry}
    
        \vspace{0.2 cm}



    \section{Courses \band Certifications}

        \begin{onecolentry}
            \begin{highlights}
   
        \item{\textbf{MySQL Explorer;} Oracle University (Sep 19, 2024)}
    
        \item{\textbf{Scrum Foundation Professional Certification - SFPC™;} CertiProf® Professional Knowledge (July 02, 2023)}
        
        \item{\textbf{Data Analysis for Health Surveillance;} UFSC: Federal University of Santa Catarina (May 15, 2023)}
    
        \item{\textbf{SGDB PostgreSQL;} IFRS: Federal Institute of Rio Grande do Sul (Apr 02, 2022)}
    
        \item{\textbf{Leadership Coaching;} Leading for High-Performance (SEBRAE) (Nov 12, 2021)}
        
        \item{\textbf{Statistical Foundations;} LabTime (CAPES) (Nov 11, 2020)}
        
        \item{\textbf{Introduction to Data Visualization with Matplotlib;} {DataCamp} (Sep 16, 2020)}
        
        \item{\textbf{Visualizing Geospatial Data in Python;} {DataCamp} (Sep 08, 2020)}
        
        \item{\textbf{Unit Testing for Data Science in Python;} {DataCamp} (Jun 10,2020)}
        
        \item{\textbf{Intermediate Importing Data in Python;} {DataCamp} (May 28, 2020)}
        
        \item{\textbf{Introduction to SQL;} {DataCamp} (May 22, 2020)}
        
        \item{\textbf{Introduction to Importing Data in Python;} {DataCamp} (May 13, 2020)}
        
        \item{\textbf{Bash Scripting;} {DataCamp} (May 20, 2020)}
        
        \item{\textbf{Conda Essentials;} {DataCamp} (May 04, 2020)}
        
        \item{\textbf{Git;} {DataCamp} (May 01, 2020)}
        
        \item{\textbf{Manipulating DataFrames with Pandas;} {DataCamp} (Jan 14, 2020)}
        
        \item{\textbf{CCNA; Network fundamentals, Network access, IP connectivity, IP services, Security fundamentals, Network Protocols, Automation and programmability;} {SENAC/CISCO} (2006)}
        
        \item{\textbf{Windows 20003Server / Concepts and Infrastructure Network / Infraestruture and Active Directory;} {TREITEC} {(2004)}}
        	
        \item{\textbf{Delphi 6 Basic;} {ProWay Informática} {(Mar 08, 2003)}}
 
            \end{highlights}
        \end{onecolentry}


        \vspace{0.2 cm}

    
    \section{Projects and Open Source Contributions}
        \vspace{0.10 cm}
        \begin{onecolentry}
            \begin{highlights}

        \item \textbf{SecondDx:} Contributed to the development of a platform that enhances diagnostic accuracy through AI-powered second opinions, focusing on patient anamnesis, medical imaging, and health prognostics.
        \item \textbf{MyMHAI - Mental Health Care:} Contributed to the development of a mental health platform focused on leveraging AI technologies to support mental health assessment and interventions. Developed data-driven tools for analyzing mental health patterns, integrated AI models for predictive analytics, and enhanced the accessibility of mental health resources through automated systems.
        \item \textbf{Literev (TheGraphNetwork):} Developed tools for scientific literature review, focusing on Retrieval-Augmented Generation (RAG) workflows for automated document retrieval and knowledge extraction. Contributed to the RAGO library, optimizing document retrieval and AI-assisted knowledge extraction with OpenAI models. Additionally, supported other data-driven projects within TheGraphNetwork, enhancing public health and scientific research applications.
        \item \textbf{InfoDengue:} Contributed to the development of data-driven tools for monitoring arboviruses in Brazil, focusing on epidemiological data analysis and predictive modeling. Supported the AlertaDengue web portal for real-time monitoring of arboviruses by integrating climate, health, and social data. Developed PySUS, a Python library for accessing and analyzing datasets from Brazil's Unified Health System (SUS), and automated climate data collection through the CrawlClima tool to support public health models. Implemented scalable storage solutions with MinIO (StorageBox-MinIO) for managing epidemiological data and designed PostgreSQL replication solutions using Bucardo (infodengue-bucardo) to ensure data consistency and high availability.
        \item \textbf{EpiGraphHub (TheGraphNetwork):} Contributed to epidemiological data platforms for public health analysis, focusing on data integration, visualization, and real-time analytics.
        \item \textbf{OpenScienceLabs:} Active contributor to open-source projects promoting data accessibility, reproducibility, and scientific research tools, supporting initiatives like Literev and other innovative data-driven applications.                    
                                
            \end{highlights}
        \end{onecolentry}

     \section{Education}
      
        \begin{twocolentry}{
            Sept 2021 – Dec 2024
        }
            \textbf{University Leonardo Da Vinci, Santa Catarina, BR}, Big Data and Analytical Intelligence
            
        \end{twocolentry}
        
        \vspace{0.10 cm}

        \begin{twocolentry}{
             Dec 2024 - In Progress
        }
            \textbf{World Quant University (WQU), New Orleans, LA}, Data Science Lab
            
        \end{twocolentry}

        \vspace{0.10 cm}    

\end{document}